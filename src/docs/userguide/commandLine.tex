%!TEX root = master.tex
\chapter{Command line} % (fold)
\label{cha:command_line}

\begin{tabular}{l|l|l}
\textbf{Option}(-) & \textbf{Long Option}(--) & \textbf{Meaning}\\ \hline
D & prop             & Sets system property of the JVM (e.g. -Dmyprop=myvalue).\\ 
I & noImports        & Disable usage of default imports for build script files.\\
P & projectProperty  & Set project property of the root project (e.g. -Pmyprop=myvalue).\\
S & noJvmTermination & Don't trigger a System.exit(0) for normal termination. Use by Gradle's unit tests.\\
b & buildfile        & Use this build file name (also for subprojects).\\
d & debug            & Log in debug mode (includes normal stacktrace).\\
e & embedded         & Use an embedded build script.\\
f & fullStacktrace   & Print out the full (very verbose) stacktrace.\\
g & gradleUserHome   & The user gradle home dir.\\
h & help             & Prints usage information.\\
i & depInfo          & Log dependency management output in info mode (Default mode is error).\\
j & depDebug         & Log dependency management output in debug mode (Default mode is error).\\
l & pluginProperties & Use this file as the plugin properties file.\\
n & nonRecursive     & Do not execute primary tasks of childprojects.\\
p & projectDir       & Use this dir instead of the current dir as the project dir.\\
q & quiet            & Log erros only.\\
s & stacktrace       & Print out the stacktrace.\\
t & tasks            & Show list of tasks.\\
u & noSearchUpwards  & Don't search in parent folders for a settings.gradle file.\\
v & version          & Prints version info.\\
\end{tabular}
\\
 
\noindent The same information is printed to the console when you execute \texttt{gradle -h}.\\
% chapter command_line (end)